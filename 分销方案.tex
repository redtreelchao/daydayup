% Created 2017-10-26 四 14:36
% Intended LaTeX compiler: pdflatex
\documentclass[11pt]{article}
\usepackage[utf8]{inputenc}
\usepackage[T1]{fontenc}
\usepackage{graphicx}
\usepackage{grffile}
\usepackage{longtable}
\usepackage{wrapfig}
\usepackage{rotating}
\usepackage[normalem]{ulem}
\usepackage{amsmath}
\usepackage{textcomp}
\usepackage{amssymb}
\usepackage{capt-of}
\usepackage{hyperref}
\author{lichao}
\date{\today}
\title{分销方案}
\hypersetup{
 pdfauthor={lichao},
 pdftitle={分销方案},
 pdfkeywords={},
 pdfsubject={},
 pdfcreator={Emacs 25.3.2 (Org mode 9.1.2)}, 
 pdflang={English}}
\begin{document}

\maketitle
\tableofcontents

\#+FILETAGS :方案:分销方案:
\section{分销方案}
\label{sec:org6251bd4}
\begin{center}
\begin{tabular}{lll}
author & date & memo\\
\hline
lichao & 2017/10/16 & init\\
 &  & \\
\end{tabular}
\end{center}
\subsection{会议沟通}
\label{sec:org274869e}
会议部门:开发/采购/财务
\section{具体内容}
\label{sec:org327c16e}
新增加一个大菜单 \textbf{分销系统}
\subsection{类别有两种}
\label{sec:org50c8e67}
\begin{itemize}
\item 由公司仓库发货的,才走此 \textbf{分销系统}
\item 直接由市场发往经销商的,暂时走原方案:钉钉流程
\end{itemize}
\subsection{分销商}
\label{sec:org4ad51ea}
分销商属性与供应商一样,只是类型不同。原来系统中获取供应商的方法,添加上类型默认是供应商。
\subsection{分销单}
\label{sec:orgeda641c}
由采购部创建分销单,作为与分销商的一次合作。
主要属性有:
\begin{itemize}
\item 分销商/回款时间/回款金额(动态计算)
\item 销售商品/分销价/分销数/已分销数
\item 关联分销出库单
\item 在分销单详情中,要列出出库单的状态。
\begin{itemize}
\item 单号,创建时间,是否出库,出库时间
\item 可添加/删除对应的出库单,此操作要更新关联出库单的商品数到已分销商品数中。
\end{itemize}
\item 分销单的出库状态
\begin{itemize}
\item 已删除(没有关联出库单时,可以删除)
\item 待出库
\item 部分出库(有已分销数,但不全一致时)
\item 待出库结束(商品数量一致,但出库状态是未出库状态)。以下情况会引起此状态的变化
\begin{itemize}
\item 人工修改分销数
\item 自动创建分销出库单时
\item 添加/删除出库单时
\end{itemize}
\item 已出库(扫描 \textbf{待出库结束} 状态的分销单,若所有出库单均已出库时,变成此状态)
\end{itemize}
\item 分销单的结算状态( 已出库状态的分销单,可进入结算流程)
\begin{itemize}
\item 待出批次
\item 已生批次
\end{itemize}
\end{itemize}

\subsection{分销出库单}
\label{sec:org0e3e50d}
在分销单列表上,针对未完结的分销单,可生成分销出库单。系统自动创建分销出库单,并与分销单关联。\\
\begin{itemize}
\item 一个分销单在未全出之前可以 多次生成分销出库单
\item 自动创建分销出库单时,要弹出页面,可修改待分销商品数(分销数-已分销数)
\item 当分销数与已分销数一致时,分销单的状态变为待出库结束
\end{itemize}
\subsection{分销商回款调整}
\label{sec:orgaa1301c}
与供应商的结算调整一致。由采购人员日常添加。(尽量使用已经有表)
\subsection{分销回款批次}
\label{sec:orgbfae181}
与采购单的结算批次一致
\begin{itemize}
\item 状态有: \textbf{未申请/已申请/已回款}
\item 可设置调整金额明细
\item 不需要开票信息等
\item 财务有权限将 \textbf{已申请} 的变成 \textbf{已回款}
\end{itemize}
\end{document}
